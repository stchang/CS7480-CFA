\documentclass{article}

\usepackage{amsfonts}
\usepackage{cs7480}

\title{CS 7480 Spring 2013 Control Flow Analysis}
\author{notes by: Stephen Chang}
\date{Friday, January 11, 2013}

\begin{document}

\maketitle

\section{Summary}
General monotone flow analysis framework.
\begin{enumerate}
  \item Monotonic, continuous functions
  \item Lattices
    \begin{itemize}
      \item represents a program property, ie, a ``fact'' lattice
      \item ordering is information ordering
      \item modeling programs as functions on lattices
    \end{itemize}
\end{enumerate}

\section{Monotonic, continuous functions}
\begin{definition}
  A \emph{least upper bound ($\lub$)} on set $S$:%, ie $\bigsqcup S$:
  \begin{enumerate}
    \item $\forall x \in S: x \leq \lub{S}$, \\
      ie, $\lub{S}$ is an upper bound of $S$
    \item $\forall x \in S: x \leq y \implies \lub{S} \leq y$,\\
      ie, $\lub{S}$ is less than or equal to all upper bounds of $S$.
  \end{enumerate}
\end{definition}

\noindent If ordering is information ordering, then $\lub{S}$ is the ``most
precise'' upper bound.

\begin{definition}
  In a \emph{complete partial order (cpo)}, every chain has a lub.
\end{definition}

\begin{definition}
  A \emph{pointed set} has a bottom ($\bot$) element.
\end{definition}

\begin{definition}
  A function $f$ is \emph{monotone} if $\forall a,b:a\leq b \implies f(a) \leq
  f(b)$. 
\end{definition}

\noindent In other words, more information gives better conclusions. But we
don't need to worry much about monotonicity because it's pretty much always
true.

\begin{definition}
  A function $f$ is \emph{continuous} if $\forall \textrm{ chains } C:
  f(\lub{C}) = \lub{f(C)}$.
\end{definition}

\noindent In other words, the function and $\lub$ operations commute. So you
``can sneak up on an answer.''

\begin{theorem} \label{thm:contmono}
  continuous $\implies$ monotone
\end{theorem}
%
\begin{proof}.
  \begin{enumerate}
  \item $x\leq y\implies \lub{\set{x,y}} = y$
  \item From the definition of continuous:
    $$f(\lub{\set{x,y}}) = \lub{f(\set{x,y})}$$
    Using step 1:
    $$f(y) = \lub{f(\set{x,y})} = \lub{\set{f(x),f(y)}}$$
    so
    $$f(x)\leq f(y)$$
    meaning $f$ is monotone.
  \end{enumerate}
\end{proof}

\begin{theorem}
  monotonic $\implies$ continuous, when cpo chains are finite
\end{theorem}
%
\begin{proof}
  Consider (finite) chain $C = x_1\leq\cdots\leq x_n$. Want to show: 
$$f(\lub{C}) = \lub{f(C)}$$
\begin{enumerate}
  \item $\lub{C} = x_n$, so $f(\lub{C}) = f(x_n)$
  \item By monotonicity:
    $$f(x_1)\leq\cdots\leq f(x_n)$$
    so
    $$\lub{f(C)} = f(x_n)$$
\end{enumerate}
\end{proof}

\begin{theorem}
  With an infinite chain, monotone $\;\;\not\!\!\!\!\implies\!$ continuous.
\end{theorem}
%
\begin{proof}
  Counterexample:\\
  $f:\mathbb{N}_\infty\rightarrow\mathbb{N}_\infty$, $f(i) = 0$, for $i \in \mathbb{N}$, $f(\infty) = 1$\\
  Consider the chain $C = \mathbb{N}$. $f(\lub{C}) = f(\infty) = 1$, but $\lub{f(C)} = 0$.
\end{proof}

\newcommand{\fnset}[1][0]{\ensuremath{\set{f^n(\bot)\mid n\geq #1}}}

\begin{theorem}[Least Fixed-point]
  If $D$ is a pointed cpo and $f:D\rightarrow D$ is continuous, then $f$ has a \emph{least fixed-point (lfp)}, $\fix{f} = \lub{\fnset}$
\end{theorem}

\noindent Recall that a recursive function is a search for a fixed point. If
there is a least fixed-point, then there is a ``best answer''.

\begin{proof}
Summary:
\begin{enumerate}
  \item Show that $\fnset$ has a $\lub$.
  \item Show that the $\lub$ is a fixed-point.
  \item Show that the $\lub$ is a least fixed-point.
\end{enumerate}
Proof:
\begin{enumerate}
  % Show that $\fnset$ has a $\lub$.
\item 
  \begin{enumerate}
    \item continuous $\implies$ monotone, so $\bot\leq f(\bot)$, since $\bot$
      is $\leq$ everything
    \item using the same reasoning, we get the chain:
      $$\bot\leq f(\bot)\leq f^2(\bot)\leq f^3(\bot)\leq\cdots$$
      Since $f^0(\bot) = \bot$, $\fnset$ is a chain.
    \item since $D$ is complete, every chain has a $\lub$ so $\fnset$ has a
      $\lub$
  \end{enumerate}
  % Show that the $\lub$ is a fixed-point.
\item 
  \begin{enumerate}
  \item Let $\lub{\fnset} = \fix{f}$. Then applying $f$ to both sides:
    $$f(\fix f)=f(\lub{\fnset})$$
  \item By continuity:
    $$f(\lub{\fnset}) = \lub{f(\fnset)} = \lub{\fnset[1]}$$
  \item $\fnset[1]$ is still a chain and removing the bottom element doesn't
    change the $\lub$, so:
    $$\lub{\fnset[1]} = \lub{\fnset} = \fix{f}$$
    So $f(\fix{f}) = \fix{f}=\lub{\fnset}$ is a fixed-point.
  \end{enumerate}
  % Show that the $\lub$ is a least fixed-point.
\item
  \begin{enumerate}
  \item Suppose $d'$ is a fixed-point so $d' = f(d')$. We know $\bot\leq d'$
    and by monotonicity, $f(\bot)\leq f(d')$.  
  \item But $f(d') = d'$ so $f(\bot)\leq d'$.
  \item Applying $f$ again $f^2(\bot)\leq f(d') = d'$ so $f^2\leq d'$. Since
    $f^n(\bot)\leq d'$, $d'$ is an upper bound of $\fnset$.
  \item Since $\fix{f}$ is the least upper bound of $\fnset$, $\fix{f}\leq
    d'$. Since $\fix{f} = \lub{\fnset}$ is also a fixed-point, it must be the
    least fixed-point.
  \end{enumerate}
\end{enumerate}
\end{proof}

\subsubsection*{Takeaway from least fixed-point theorem:}
If I can set up a problem as recursive equations, ie, a search for a
fixed-point, then if my fact space has certain properties (pointed, continuous
(and thus monotone) complete partial order), then there is a ``best'' answer.

\paragraph{Fun fact (unrelated to this course):} The $\fix$ function is also continuous.

\section{Lattices}
\begin{definition}\label{def:lattice}
  A poset $(S,\leq)$ is a \emph{lattice} iff $\forall x,y\in S$, there exists a
  unique meet $\meet{x}{y}$ and join $\join{x}{y}$ such that:
  \begin{enumerate}
    \item $\meet{x}{y}\leq x$
    \item $\meet{x}{y}\leq y$
    \item $z\leq \meet{x}{z}\leq y\implies z\leq \meet{x}{y}$
  \end{enumerate}
  In other words, the meet is the greatest lower bound, ie the infimum. Also:
  \begin{enumerate}
    \item $\join{x}{y}\geq x$
    \item $\join{x}{y}\geq y$
    \item $z\geq \join{x}{z}\geq y\implies z\geq\join{x}{y}$
  \end{enumerate}
  In other words, the join is the least upper bound, ie the supremum.
\end{definition}

\paragraph{Other kinds of lattices:}
\begin{itemize}
  \item A semi-lattice has either a supremum or infimum but not both.
  \item A bounded lattice has a $\bot$ and $\top$ element.
  \item In a complete lattice, every set has a meet and join, including
    infinite sets.
  \item So complete $\implies$ bounded and finite lattices are complete.
\end{itemize}

\begin{definition}[Alternative (algebraic) lattice definition]
  A \emph{lattice} is a set with meet and join operations such that they are:
  \begin{itemize}
    \item commutative
    \item associative
    \item absorption (ie $\join{x}{(\meet{x}{y})} = x$ and $\meet{x}{(\join{x}{y}}) = x$)
  \end{itemize}
  With these properties, we can prove the conditions from
  definition~\ref{def:lattice}.
\end{definition}

\paragraph{meet and join are idempotent:}
\begin{itemize}
  \item $\join{x}{x} = x$
  \item $\meet{x}{x} = x$
  \item $\join{x}{\bot} = x$
  \item $\meet{x}{\top} = x$
\end{itemize}

\paragraph{(Unrelated) homework: } Use the word ``palimpsest'' in a conversation.
\paragraph{Bonus:} Use the word ``boustrophedon'' in a conversation.

\paragraph{Non-guaranteed properties of lattices:}
\begin{itemize}
  \item distributivity: $\join{x}{(\meet{y}{z})} = \meet{(\join{x}{y})}{(\join{x}{z})}$
  \item well-foundedness: every chain is finite, ie may be able to compute it
\end{itemize}

\paragraph{Some concrete lattices:}
\begin{itemize}
  \item powerset of $S$, $\mathcal{P}(S)$ is a lattice, $\leq = \subseteq$
    (subset), meet = $\cap$, join = $\cup$, $\bot = \set{}$, $\top = S$
  \item integers $\mathbb{Z}$ is an unbounded lattice, meet = min, join = max
  \item $\mathbb{Z}_{-\infty,\infty}$ is a bounded lattice (I forgot the
    analysis examples that use this lattice)
  \item $\mathbb{Z}_{\bot,\top}$ is a flat lattice
  \item logicians use the lattice of propositions: meet = and, join = or, $\leq
    = \implies$
  \item range analysis lattice $\mathcal{P}(Z)$: represent sets of integers as
    ranges $[i,j]$

    examples: 

    $\join{[0,7]}{[5,10]} = [0,10]$

    $\join{[0,5]}{[10,15]} = [0,15]$ (some precision was lost)
\end{itemize}

In flow analysis, there are 2 sources of ``crap'' (imprecision)
\begin{enumerate}
  \item join operation introduces approximation
  \item (for some reason I didnt write down a \#2)
\end{enumerate}
\end{document}
