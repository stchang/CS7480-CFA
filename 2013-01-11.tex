\documentclass{article}

\usepackage{amsfonts}
\usepackage{cs7480}

\title{CS 7480 Spring 2013 Control Flow Analysis}
\author{notes by: Stephen Chang}
\date{Friday, January 11, 2013}

\begin{document}

\maketitle

\section{Summary}
General monotone flow analysis framework.
\begin{enumerate}
  \item Monotonic, continuous functions
  \item Lattices
    \begin{itemize}
      \item represents a program property, ie, a ``fact'' lattice
      \item ordering is information ordering
      \item modeling programs as functions on lattices
    \end{itemize}
\end{enumerate}

\section{Monotonic, continuous functions}
\begin{definition}
  A \emph{least upper bound ($\lub$)} on set $S$:%, ie $\bigsqcup S$:
  \begin{enumerate}
    \item $\forall x \in S: x \leq \lub{S}$, \\
      ie, $\lub{S}$ is an upper bound of $S$
    \item $\forall x \in S: x \leq y \implies \lub{S} \leq y$,\\
      ie, $\lub{S}$ is less than or equal to all upper bounds of $S$.
  \end{enumerate}
\end{definition}

\noindent If ordering is information ordering, then $\lub{S}$ is the ``most
precise'' upper bound.

\begin{definition}
  In a \emph{complete partial order (cpo)}, every chain has a lub.
\end{definition}

\begin{definition}
  A \emph{pointed set} has a bottom ($\bot$) element.
\end{definition}

\begin{definition}
  A function $f$ is \emph{monotone} if $\forall a,b:a\leq b \implies f(a) \leq
  f(b)$. 
\end{definition}

\noindent In other words, more information gives better conclusions. But we
don't need to worry much about monotonicity because it's pretty much always
true.

\begin{definition}
  A function $f$ is \emph{continuous} if $\forall \textrm{ chains } C:
  f(\lub{C}) = \lub{f(C)}$.
\end{definition}

\noindent In other words, the function and $\lub$ operations commute. So you
``can sneak up on an answer.''

\begin{theorem} \label{thm:contmono}
  continuous $\implies$ monotone
\end{theorem}
%
\begin{proof}.
  \begin{enumerate}
  \item $x\leq y\implies \lub{\set{x,y}} = y$
  \item From the definition of continuous:
    $$f(\lub{\set{x,y}}) = \lub{f(\set{x,y})}$$
    Using step 1:
    $$f(y) = \lub{f(\set{x,y})} = \lub{\set{f(x),f(y)}}$$
    so
    $$f(x)\leq f(y)$$
    meaning $f$ is monotone.
  \end{enumerate}
\end{proof}

\begin{theorem}
  monotonic $\implies$ continuous, when cpo chains are finite
\end{theorem}
%
\begin{proof}
  Consider (finite) chain $C = x_1\leq\cdots\leq x_n$. Want to show: 
$$f(\lub{C}) = \lub{f(C)}$$
\begin{enumerate}
  \item $\lub{C} = x_n$, so $f(\lub{C}) = f(x_n)$
  \item By monotonicity:
    $$f(x_1)\leq\cdots\leq f(x_n)$$
    so
    $$\lub{f(C)} = f(x_n)$$
\end{enumerate}
\end{proof}

\begin{theorem}
  With an infinite chain, monotone $\;\;\not\!\!\!\!\implies\!$ continuous.
\end{theorem}
%
\begin{proof}
  Counterexample:\\
  $f:\mathbb{N}_\infty\rightarrow\mathbb{N}_\infty$, $f(i) = 0$, for $i \in \mathbb{N}$, $f(\infty) = 1$\\
  Consider the chain $C = \mathbb{N}$. $f(\lub{C}) = f(\infty) = 1$, but $\lub{f(C)} = 0$.
\end{proof}

\begin{theorem}[Least Fixed-point]
  If $D$ is a pointed cpo and $f:D\rightarrow D$ is continuous, then $f$ has a \emph{least fixed-point (lfp)}, $\fix{f} = \lub{\set{f^n\bot\mid n\geq 0}}$.
\end{theorem}

\noindent Recall that a recursive function is a search for a fixed point. If
there is a least fixed-point, then there is a ``best answer''.

\begin{proof}
Summary:
\begin{enumerate}
  \item Show that $\set{f^n\bot\mid n\geq 0}$ has a $\lub$.
  \item Show that the $\lub$ is a fixed-point.
  \item Show that the $\lub$ is a least fixed-point.
\end{enumerate}
Proof:
\begin{enumerate}
\item
\item
\item
\end{enumerate}
\end{proof}

\end{document}
